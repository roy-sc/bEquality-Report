\paragraph*{Strengths}
-Our solution provides a reliable, transparent and automatic way of creating a gender-equality index. Compared to other already existing indexes we offer an index where the reliability and correctness can be verified by each and every person. This is done by implementing our strategy on the block-chain. 
-Our approach allows everybody to classify companies according to the latest law regulatory for gender-equality in companies.\\
%strengths/what is working in our solution
-Our solution is multidimensional, where multidimensional stands for two different things. First of all our index contains different categories where the result is weighted and then demonstrated as a graph [graph bild]. The other multidimensionality of our project is that we collect data from two different sources. We collect it from the managers and we collect the data from a casual worker as well.
-public\\
-cheap\\ Nowadays these indexes require a lot of work e.g. to collect the data, to evaluate the data, to present the data in a meaningful way and so on. In our solution we are trying to do as much as possible in an automatic approach.

\paragraph*{Weaknesses}
- Each participant needs to create his own blockchain account before taking the survey\\
- One has to handle a large chunk of data.

\paragraph*{Open Challenges}
There are several ways to expand our project. Examples would be
-to achieve a full working automatic way of indexing gender-equality. This is our main idea but we haven't completed the implementation. The application we use to do the survey has just an interface. This means that we should collect the data on the application and try to put the collected data onto the block-chain for further calculation processes. 
-E-voting system should be improved\\

	
\paragraph*{Disruptional Potential}
-transparency\\
-further enhance gender equality