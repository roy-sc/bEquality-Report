There are still some challenges to solve for this project.\\[5pt]
One major challenge is to provide a secure way of implementing a system equivalent to E-voting, i.e. to find a system where a user applies to a survey or evaluation and the data provided by him cannot be traced back to this user, even when the user can verify that the data provided by him is used and not some modified version of his data.\\
A way of solving this problem for this use-case (of a gender-equality index) it may be a solution to implement the system in such a way, that cheating for companies are too expensive to be profitable and the company therefore acts cooperatively and provides true data.\\[5pt]
Once we are certain that the company does not cheat we have to look at the behaviour of those who can have a great impact on the evaluation: the employees.\\
We have to assume that the participation rate of the employee survey is rather low, surely not more than 50 percent. This is because people who are moderately happy often do not feel like they have to praise their employer, but they also do not complain. Therefore, most of the people who will respond are either unhappy or very happy, the votes of the moderate people get lost. To get a useful a participation rate, this means we have to send emails to every employee.\\
We have to be certain that with our questions, we get the information we really want. They have to be detailed and unambiguous. Furthermore, we have to adapt the questions for different hierarchy levels. A manager may have other concerns than a worker.
To not lose the participator?s interest, we have to make sure that the questions are short, easy to understand and that there are not too many of them.\\
Also, it is our concern that there is no (easy) way to identify the voter by his answer. A rather simple way to avoid part of it is the strategy of randomised response. This means that if two people are at the exact same point in the survey, let?s say they both have two choices to tick, one may be asked a trivial question, like which sort of ice cream he likes. The other one will be asked a serious question which is important for out data collection. This way, even if someone knows at which point which choice has been made, (for example he sees the employee ticking on his smartphone, but cannot read the question from a distance) he does not know what it meant.\\
We know that anonymising data opens the door for abusers who try to manipulate the survey. The more anonymous the participation is, the easier it is to abuse it. Sadly we have not yet found a way to fix this completely - we can only agree on a certain balance between control and anonymity.\\

	
\comment{
challenges for implementation\\		
-Finding a secure way of implementing a system equivalent to E-voting.\\ 
-Making cheating for companies too hard to be profitable.\\
-privacy for gathering data, boss should not have possibility to see results from its employees\\
-costs\\
Lena: I've added the things I learned from the interview with Prof. Renate Schubert in this section\\
But how do i fix this: participator?s\\

done???
}